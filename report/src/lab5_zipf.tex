\section{Закон Ципфа}

\subsection{Теоретические основы}

\textbf{Закон Ципфа} утверждает, что частота слова в корпусе текстов обратно пропорциональна его рангу:

\[ f(r) = \frac{C}{r} \]

где $f(r)$ --- частота термина с рангом $r$, $C$ --- константа, зависящая от корпуса.

В логарифмической шкале закон Ципфа представляет собой прямую линию с наклоном $-1$:

\[ \log f(r) = \log C - \log r \]

Более точное приближение даёт обобщённый закон Ципфа--Мандельброта:
\[ f(r) = \frac{C}{(r + b)^a} \]
где $a$ и $b$ --- параметры, подбираемые по реальным данным.

\subsection{Реализация анализатора}

Класс \texttt{ZipfAnalyzer} реализован на C++ и выполняет следующие функции:
\begin{enumerate}
    \item Подсчёт частот всех терминов после стемминга с использованием собственной хеш-таблицы \texttt{StringMap}
    \item Сортировка терминов по убыванию частоты (сортировка слиянием, $O(N \log N)$)
    \item Присвоение рангов
    \item Передача данных через HTTP API для визуализации во фронтенде
\end{enumerate}

Данные для построения графиков передаются через API-эндпоинт \texttt{/api/zipf}, который возвращает JSON с рангами, частотами и теоретическими предсказаниями. Визуализация выполняется в веб-интерфейсе с помощью библиотеки Chart.js.

\subsection{Визуализация}

На рис.~\ref{fig:zipf_linear} представлено распределение в линейной шкале для топ-1000 терминов. Линейная шкала наглядно показывает резкий спад частот: самый частый термин (<<на>>) имеет частоту $\sim$1\,220\,000, тогда как теоретическая кривая Ципфа даёт для ранга 1 значение примерно в 6 раз ниже. Обе кривые быстро сходятся к оси абсцисс, и для рангов выше 200--300 различия становятся визуально неразличимы --- типичное ограничение линейной шкалы при анализе закона Ципфа.

\begin{figure}[H]
\centering
\includegraphics[width=0.9\textwidth]{images/linear_zipf.png}
\caption{Закон Ципфа: линейная шкала (топ 1000). Реальные данные (синяя линия) и теоретическая кривая (красная пунктирная).}
\label{fig:zipf_linear}
\end{figure}

На рис.~\ref{fig:zipf_log} и~\ref{fig:zipf_loglog} представлены графики в логарифмической шкале (log--log), где закон Ципфа изображается прямой линией с наклоном $-1$. Реальные данные (синие точки) в целом следуют линейному тренду, но систематически лежат \emph{выше} теоретической прямой: на малых рангах расхождение наибольшее, в области средних и больших рангов точки образуют кривую, примерно параллельную теоретической линии, но смещённую вверх. Это согласуется с превышением частот в голове распределения (топ-термины встречаются чаще, чем предсказывает классический закон Ципфа). Аппроксимация имеет наклон -0.716

\begin{figure}[H]
\centering
\includegraphics[width=0.9\textwidth]{images/log_zipf.png}
\caption{Закон Ципфа: логарифмическая шкала. Сравнение реальных данных с теоретической прямой.}
\label{fig:zipf_log}
\end{figure}

\begin{figure}[H]
\centering
\includegraphics[width=0.9\textwidth]{images/log_log_zipf.png}
\caption{Закон Ципфа: логарифмическая шкала (log--log). Реальные данные и теоретическая прямая $\log f = \log C - \log r$.}
\label{fig:zipf_loglog}
\end{figure}

\begin{figure}[H]
\centering
\includegraphics[width=0.9\textwidth]{images/top_terms.png}
\caption{Топ-30 наиболее частых терминов.}
\label{fig:top_terms}
\end{figure}


\subsection{Причины расхождения с теоретической моделью}

Реальное распределение отклоняется от идеального закона Ципфа:

\begin{enumerate}
    \item \textbf{Голова распределения}: самые частые слова (стоп-слова, предлоги) встречаются чаще, чем предсказывает закон. Это объясняется тем, что служебные слова используются в каждом предложении.
    
    \item \textbf{Хвост распределения}: редкие термины (специализированная терминология, названия языков) более многочисленны, чем предсказано. Лингвистический корпус содержит множество уникальных терминов.
    
    \item \textbf{Эффект стемминга}: объединение словоформ увеличивает частоты базовых форм и смещает распределение.
    
    \item \textbf{Предварительная обработка}: фильтрация коротких токенов и чисел влияет на хвост распределения.
\end{enumerate}

\subsection{Журнал выполнения задания}

При реализации анализатора закона Ципфа были выявлены следующие проблемы и их решения:

\begin{enumerate}
    \item \textbf{Выбор библиотеки для визуализации}: Изначально планировалось использовать серверную генерацию графиков (например, matplotlib через Python), но это требовало дополнительной инфраструктуры и замедляло отображение результатов. Решение: передача данных через HTTP API в формате JSON и визуализация на клиенте с помощью Chart.js.
    
    \item \textbf{Обработка длинного хвоста распределения}: Распределение содержит $\sim$940\,000 уникальных терминов, из которых $\sim$70--80\% встречаются редко (частота $\leq$ 5). Отображение всех точек на графике создавало визуальный шум и затрудняло анализ. Решение: агрегация редких терминов по диапазонам рангов и отображение только репрезентативных точек для хвоста распределения.

\end{enumerate}

\subsection{Выводы}

Анализ распределения терминов в корпусе подтверждает применимость закона Ципфа к реальным текстовым данным, хотя наблюдаются систематические отклонения от теоретической модели.

\textbf{Подтверждение закона Ципфа:}
Распределение частот терминов в целом следует закону Ципфа: наиболее частые слова (<<на>>, <<год>>, <<был>> и др.) занимают первые ранги, а редкие термины образуют длинный хвост. На графике log--log реальные данные образуют приблизительно линейный тренд, параллельный теоретической прямой, что подтверждает степенной характер зависимости. Систематическое смещение реальных точек вверх относительно теоретической линии отражает превышение частот в голове распределения.

\textbf{Систематические отклонения:}
\begin{itemize}
    \item \textbf{Голова распределения}: Превышение частот для топ-10--20 терминов объясняется тематической спецификой корпуса (термины <<язык>>, <<слово>>, <<термин>> встречаются в каждой статье) и эффектом стемминга, объединяющего словоформы.
    
    \item \textbf{Хвост распределения}: Большое количество редких терминов характерно для лингвистического корпуса, содержащего множество специализированной терминологии и названий языков.
\end{itemize}

\textbf{Влияние на поисковую систему:}
\begin{itemize}
    \item Высокая концентрация частот в топ-100 терминах (~35--40\% вхождений) означает, что инвертированный индекс для этих терминов будет очень большим, что влияет на производительность поиска
    \item Длинный хвост редких терминов требует эффективной структуры данных для хранения постинговых списков с низкой частотой
    \item Закон Ципфа подтверждает необходимость стоп-слов для фильтрации самых частых, но малосодержательных терминов
\end{itemize}

\textbf{Недостатки текущего анализа:}
\begin{itemize}
    \item Анализ выполнен только для стеммированных терминов --- распределение исходных словоформ может отличаться
    \item Не учитывается влияние длины документа на частоту терминов (длинные статьи могут искажать распределение)
    \item Отсутствует сравнение с другими корпусами для выявления специфики лингвистической тематики
\end{itemize}

\textbf{Возможные улучшения:}
\begin{itemize}
    \item Сравнение распределения до и после стемминга для оценки влияния нормализации
    \item Анализ распределения по частям речи (существительные, прилагательные, глаголы) для более глубокого понимания структуры корпуса
    \item Применение обобщённого закона Ципфа--Мандельброта для более точного моделирования отклонений
    \item Визуализация распределения для отдельных категорий корпуса (например, только статьи о фонетике или только о синтаксисе)
\end{itemize}

\pagebreak
