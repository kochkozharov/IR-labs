\section{Булев поиск}

\subsection{Операторы булева поиска}

Реализованы три основных оператора:

\begin{itemize}
    \item \textbf{AND} --- пересечение множеств документов (документ должен содержать все термины)
    \item \textbf{OR} --- объединение множеств документов (документ должен содержать хотя бы один термин)
    \item \textbf{NOT} --- разность множеств (исключение документов, содержащих термин)
\end{itemize}

По умолчанию между терминами подразумевается оператор AND.

\subsection{Алгоритмы операций над множествами}

Все операции реализованы на отсортированных списках документов с линейной сложностью $O(n + m)$.

\textbf{Пересечение (AND):}
\begin{enumerate}
    \item Два указателя движутся по отсортированным posting lists
    \item При совпадении элементов --- добавление в результат и продвижение обоих указателей
    \item При несовпадении --- продвижение указателя на меньшем элементе
\end{enumerate}

\textbf{Объединение (OR):}
\begin{enumerate}
    \item Слияние двух отсортированных списков
    \item При совпадении --- добавление одного экземпляра и продвижение обоих
    \item При несовпадении --- добавление меньшего и продвижение его указателя
\end{enumerate}

\textbf{Разность (NOT):}
\begin{enumerate}
    \item Последовательный проход по первому списку
    \item Если элемент присутствует во втором списке --- пропуск
    \item Иначе --- добавление в результат
\end{enumerate}

\subsection{Парсинг запросов}

Анализатор запросов реализован в классе \texttt{BooleanSearch}:
\begin{enumerate}
    \item Строка запроса разбивается по пробелам
    \item Распознаются ключевые слова AND, OR, NOT
    \item Остальные слова считаются поисковыми терминами
    \item Каждому термину присваивается оператор, стоящий перед ним
    \item Термины проходят стемминг перед поиском в индексе
    \item По умолчанию используется оператор AND
\end{enumerate}

\subsection{Ранжирование результатов}

Реализовано ранжирование по сумме частот терминов запроса в документе (TF-схема):
\begin{enumerate}
    \item Для каждого документа из результата вычисляется score
    \item Score $= \sum_{t \in Q} \text{tf}(t, d)$, где $Q$ --- множество терминов запроса, $\text{tf}(t, d)$ --- частота термина $t$ в документе $d$
    \item Документы сортируются по убыванию score
\end{enumerate}

Достоинства: простота и скорость вычисления. Недостатки: не учитывается длина документа и редкость термина (IDF).

\subsection{Примеры запросов}

\begin{table}[H]
\centering
\begin{tabular}{lll}
\toprule
\textbf{Запрос} & \textbf{Интерпретация} & \textbf{$\sim$Результатов} \\
\midrule
роман & роман (простой) & $\sim$20\,000 \\
поэзия & поэз (стемминг) & $\sim$5\,000 \\
проза AND рассказ & проз AND рассказ & $\sim$1\,000 \\
литература OR поэзия & литератур OR поэз & $\sim$15\,000 \\
роман NOT детектив & роман NOT детектив & $\sim$18\,000 \\
автор AND произведение & автор AND произведен & $\sim$8\,000 \\
\bottomrule
\end{tabular}
\caption{Примеры булевых запросов}
\end{table}

\subsection{Производительность поиска}

\begin{itemize}
    \item Время ответа на простой запрос (1 термин): $< 10$ мс
    \item Время ответа на сложный булев запрос: $< 50$ мс
    \item Пропускная способность HTTP API: $> 100$ запросов/сек
\end{itemize}

\subsection{Веб-интерфейс}

Веб-интерфейс реализован на Python (Flask) и взаимодействует с C++ движком через HTTP API. Интерфейс предоставляет:
\begin{itemize}
    \item Поисковую форму с поддержкой булевых операторов
    \item Отображение результатов с заголовками, URL и сниппетами
    \item Постраничную навигацию
    \item Вкладку с визуализацией закона Ципфа (Chart.js)
    \item Вкладку со статистикой индекса
\end{itemize}

API-эндпоинты движка:
\begin{itemize}
    \item \texttt{GET /api/search?q=...} --- булев поиск
    \item \texttt{GET /api/stats} --- статистика индекса
    \item \texttt{GET /api/zipf?limit=N} --- данные для графика Ципфа
    \item \texttt{GET /api/document?url=...} --- получение документа
\end{itemize}

\subsection{Утилита командной строки}

Движок поддерживает интерактивный режим CLI:
\begin{verbatim}
> синтаксис AND морфология
Found 523 results (12.4 ms):
1. https://ru.wikipedia.org/.../Синтаксис (score: 47)
2. https://ru.wikipedia.org/.../Морфология (score: 38)
...
\end{verbatim}

\subsection{Демонстрация веб-интерфейса}

% TODO: Вставить скриншоты веб-интерфейса
\begin{center}
\textit{[Место для скриншота веб-интерфейса --- поиск]}

% Раскомментировать после добавления изображения:
% \begin{figure}[H]
% \centering
% \includegraphics[width=\textwidth]{images/ui_search.png}
% \caption{Веб-интерфейс: результаты поиска}
% \label{fig:ui_search}
% \end{figure}
\end{center}

\begin{center}
\textit{[Место для скриншота веб-интерфейса --- статистика]}

% Раскомментировать после добавления изображения:
% \begin{figure}[H]
% \centering
% \includegraphics[width=\textwidth]{images/ui_stats.png}
% \caption{Веб-интерфейс: статистика индекса}
% \label{fig:ui_stats}
% \end{figure}
\end{center}

\subsection{Контейнеризация и запуск}

Все компоненты упакованы в Docker-контейнеры и управляются через \texttt{docker-compose}:

\begin{verbatim}
# Сбор корпуса (однократно)
docker-compose up scraper

# Запуск движка и веб-интерфейса
docker-compose up engine frontend
\end{verbatim}

Веб-интерфейс доступен по адресу \texttt{http://localhost:8080}.

\pagebreak
