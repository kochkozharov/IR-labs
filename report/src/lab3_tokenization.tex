\section{Токенизация}

\subsection{Определение и цели}

\textbf{Токенизация} --- процесс разбиения текста на элементарные единицы (токены), которые используются для построения поискового индекса. Токен --- последовательность символов, представляющая слово или другую значимую лексическую единицу.

Цели токенизации:
\begin{itemize}
    \item Выделение значимых единиц текста из потока символов
    \item Нормализация: приведение к единообразному виду (lowercase)
    \item Фильтрация шума: удаление HTML-артефактов, служебных символов
    \item Формирование словаря для индексации
\end{itemize}

\subsection{Реализация}

Токенизатор реализован в виде класса \texttt{Tokenizer} на C++. Основная функция \texttt{tokenize} принимает строку текста и возвращает вектор токенов. Каждый токен содержит текст и позицию в исходном документе.

Алгоритм работы (однопроходный, сложность $O(n)$):
\begin{enumerate}
    \item Последовательное чтение символов из входной строки
    \item Определение типа символа: кириллический (UTF-8, двухбайтовый), латинский, цифра или разделитель
    \item Накопление символов в буфере до встречи разделителя
    \item При встрече разделителя --- проверка валидности токена
    \item Если токен валиден --- добавление в результирующий список
\end{enumerate}

\subsection{Правила токенизации}

\subsubsection{Разделители}

\begin{itemize}
    \item Пробельные символы: пробел, табуляция, перевод строки
    \item Знаки препинания: точка, запятая, восклицательный и вопросительный знаки, точка с запятой, двоеточие
    \item Скобки: круглые, квадратные, фигурные
    \item Кавычки: одинарные и двойные
    \item Специальные символы: угловые скобки, слеши, амперсанды
\end{itemize}

\subsubsection{Обработка UTF-8 кириллицы}

Кириллические символы в UTF-8 представлены двухбайтовыми последовательностями. Токенизатор проверяет первый байт (0xD0 или 0xD1) и читает следующий байт для формирования полного символа. Приведение к нижнему регистру выполняется на уровне байтов UTF-8.

\subsubsection{Фильтрация токенов}

\begin{enumerate}
    \item Токены короче 2 символов отбрасываются
    \item Последовательности, состоящие только из цифр, удаляются
    \item Все символы приводятся к нижнему регистру
\end{enumerate}

\subsection{Достоинства метода}

\begin{enumerate}
    \item \textbf{Производительность}: однопроходный алгоритм, $O(n)$ по длине текста
    \item \textbf{Корректная работа с UTF-8}: побайтовый анализ кириллицы без использования сторонних библиотек
    \item \textbf{Низкое потребление памяти}: потоковая обработка без загрузки всего корпуса
    \item \textbf{Детерминированность}: одинаковый результат при повторной обработке
    \item \textbf{Универсальность}: работает с русским и английским языками
\end{enumerate}

\subsection{Недостатки метода}

\begin{enumerate}
    \item \textbf{Отсутствие контекстного анализа}: слова-омонимы не различаются
    \item \textbf{Проблемы с составными словами}: <<военно-морской>> разбивается на <<военно>> и <<морской>>
    \item \textbf{Аббревиатуры}: <<т.е.>>, <<и т.д.>> разбиваются по точкам
    \item \textbf{Числовые конструкции}: <<1945 год>> --- число отфильтровывается
    \item \textbf{Имена собственные}: <<Лев Толстой>> обрабатывается как два отдельных токена
\end{enumerate}

\subsection{Примеры проблемных случаев}

\begin{table}[H]
\centering
\begin{tabular}{lll}
\toprule
\textbf{Исходный текст} & \textbf{Результат} & \textbf{Проблема} \\
\midrule
<<Лев Толстой>> & [<<лев>>, <<толстой>>] & Разбиение имени \\
<<т.е.>> & [<<те>>] & Точки удалены \\
<<1869 год>> & [<<год>>] & Число отфильтровано \\
<<научно-фантастический>> & [<<научно>>, <<фантастический>>] & Дефис как разделитель \\
<<XIX>> & [] & Римские цифры удалены \\
<<non-fiction>> & [<<non>>, <<fiction>>] & Дефис разбил термин \\
\bottomrule
\end{tabular}
\caption{Проблемные случаи токенизации}
\end{table}

\pagebreak
