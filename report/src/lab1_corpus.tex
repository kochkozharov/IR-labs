\section{Добыча корпуса документов}

\subsection{Выбор тематики}

В качестве тематики корпуса выбрана \textbf{Лингвистика} --- наука о языке, его структуре, функционировании и развитии. Тема охватывает множество подкатегорий: фонетика, морфология, синтаксис, семантика, лексикология, прикладная лингвистика, описания конкретных языков и языковых семей.

Русскоязычный раздел Википедии содержит обширную коллекцию статей по данной тематике, что позволяет собрать корпус требуемого объёма (более 30\,000 документов).

\subsection{Источник данных}

\textbf{Источник:} русскоязычный раздел Википедии (\url{https://ru.wikipedia.org}).

\textbf{Начальные категории для обхода:}
\begin{itemize}
    \item Лингвистика
    \item Языки
    \item Языкознание
    \item Фонетика
    \item Морфология (лингвистика)
    \item Синтаксис
    \item Семантика
    \item Лексикология
    \item Грамматика
    \item Прикладная лингвистика
\end{itemize}

\subsection{Характеристики корпуса}

\begin{table}[H]
\centering
\begin{tabular}{lr}
\toprule
\textbf{Параметр} & \textbf{Значение} \\
\midrule
Количество документов & $\sim$35\,000 \\
Формат хранения & NDJSON \\
Минимальный размер документа & 1\,500 символов \\
Минимальное число параграфов & 3 \\
Язык & русский (с включениями на других языках) \\
Размер корпуса на диске & $\sim$500 МБ \\
\bottomrule
\end{tabular}
\caption{Характеристики корпуса документов}
\end{table}

Каждый документ содержит следующие поля:
\begin{itemize}
    \item \texttt{url} --- адрес страницы Википедии
    \item \texttt{title} --- заголовок статьи
    \item \texttt{text} --- очищенный текст статьи
    \item \texttt{word\_count} --- количество слов
    \item \texttt{paragraph\_count} --- количество параграфов
\end{itemize}

\subsection{Существующие поисковики для данного корпуса}

\begin{enumerate}
    \item \textbf{Встроенный поиск Википедии} --- ограничен только Википедией, нет возможности настраивать алгоритмы ранжирования, отсутствует явный булев поиск с операторами.
    
    \item \textbf{Google (site:ru.wikipedia.org)} --- высокое качество ранжирования, но закрытая система, невозможно изучить внутреннюю работу и алгоритмы.
    
    \item \textbf{Яндекс (site:ru.wikipedia.org)} --- аналогично Google, проприетарная реализация.
\end{enumerate}

\subsection{Недостатки существующих решений}

\begin{itemize}
    \item Невозможность настройки и изучения алгоритмов ранжирования
    \item Отсутствие булевого поиска с явными операторами AND, OR, NOT
    \item Невозможность проведения лингвистического анализа корпуса (закон Ципфа, частотный анализ)
    \item Закрытость реализации для учебных и исследовательских целей
    \item Ограниченный контроль над процессом токенизации и стемминга
\end{itemize}

\pagebreak
