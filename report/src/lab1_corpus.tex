\section{Добыча корпуса документов}

\subsection{Выбор тематики}

В качестве тематики корпуса выбрана \textbf{Лингвистика} --- наука о языке, охватывающая фонетику, фонологию, морфологию, синтаксис, семантику, лексикологию, грамматику, прикладную и компьютерную лингвистику, а также смежные области (психолингвистика, социолингвистика, этимология, письменность и др.).

Для формирования корпуса используются два источника: русскоязычный раздел Википедии ($\sim$30\,000 статей) и научная электронная библиотека КиберЛенинка ($\sim$5\,000 статей по разделу <<Языкознание и литературоведение>>). Итого корпус содержит $\sim$35\,000 документов.

\subsection{Источники данных}

\textbf{Источник 1:} русскоязычный раздел Википедии (\url{https://ru.wikipedia.org}).

\textbf{Начальные категории для обхода} (из \texttt{scraper/config.yaml}):
\begin{itemize}
    \item Лингвистика
    \item Языки, Языкознание
    \item Фонетика, Фонология
    \item Морфология (лингвистика), Синтаксис
    \item Семантика, Лексикология
    \item Грамматика, Прикладная лингвистика
    \item Психолингвистика, Социолингвистика
    \item Когнитивная лингвистика, Лингвистическая типология
    \item Диалектология, Стилистика
    \item Этимология, Орфография
    \item Письменность, Переводоведение
    \item Терминология, Компьютерная лингвистика
\end{itemize}

\textbf{Источник 2:} научная электронная библиотека КиберЛенинка (\url{https://cyberleninka.ru}).

Статьи загружаются из раздела каталога <<Языкознание и литературоведение>> (\texttt{/article/c/languages-and-literature}). Скрапер последовательно обходит страницы каталога, загружает полные тексты научных статей и фильтрует их по минимальной длине (1\,000 слов). Результат сохраняется в \texttt{corpus2.ndjson}.

\subsection{Характеристики корпуса}

\begin{table}[H]
\centering
\begin{tabular}{lr}
\toprule
\textbf{Параметр} & \textbf{Значение} \\
\midrule
Документов из Википедии & $\sim$30\,000 \\
Документов из КиберЛенинки & $\sim$5\,000 \\
Итого документов & $\sim$35\,000 \\
Формат хранения & NDJSON (\texttt{corpus.ndjson} + \texttt{corpus2.ndjson}) \\
Минимальный размер документа & 1000 слов \\
Язык & русский (с включениями на других языках) \\
\bottomrule
\end{tabular}
\caption{Характеристики корпуса документов}
\end{table}

Каждый документ (из обоих источников) содержит следующие поля:
\begin{itemize}
    \item \texttt{url} --- адрес страницы (Википедия или КиберЛенинка)
    \item \texttt{title} --- заголовок статьи
    \item \texttt{text} --- очищенный текст статьи
    \item \texttt{word\_count} --- количество слов
    \item \texttt{paragraph\_count} --- количество параграфов
\end{itemize}

Оба корпуса загружаются в единый индекс движком поиска.

\subsection{Существующие поисковики для данного корпуса}

\begin{enumerate}
    \item \textbf{Встроенный поиск Википедии} --- ограничен только Википедией, нет возможности настраивать алгоритмы ранжирования, отсутствует явный булев поиск с операторами.
    
    \item \textbf{Поиск КиберЛенинки} --- полнотекстовый поиск по научным статьям, но отсутствует булев поиск с явными операторами, невозможно объединить с другими источниками.
    
    \item \textbf{Google (site:ru.wikipedia.org)} --- высокое качество ранжирования, но закрытая система, невозможно изучить внутреннюю работу и алгоритмы.
    
    \item \textbf{Яндекс (site:ru.wikipedia.org)} --- аналогично Google, проприетарная реализация.
\end{enumerate}

\subsection{Недостатки существующих решений}

\begin{itemize}
    \item Невозможность настройки и изучения алгоритмов ранжирования
    \item Отсутствие булевого поиска с явными операторами AND, OR, NOT
    \item Невозможность проведения статистического анализа корпуса (закон Ципфа, частотный анализ)
    \item Закрытость реализации для учебных и исследовательских целей
    \item Ограниченный контроль над процессом токенизации и стемминга
\end{itemize}

\subsection{Журнал выполнения задания}

При выполнении задания по сбору корпуса документов были выявлены следующие проблемы и их решения:

\begin{enumerate}
    \item \textbf{Выбор подходящего объёма тематики}: Изначально рассматривались более узкие темы (например, только <<Фонетика>> или <<Морфология>>), но они не обеспечивали требуемый минимум в 30\,000 документов. Расширение до общей категории <<Лингвистика>> с включением подкатегорий (языки, фонетика, синтаксис, семантика и др.) позволило достичь целевого объёма.
    
    \item \textbf{Категории Википедии: слишком широкие или узкие}: Некоторые категории (например, <<Лингвистика>>) содержат сотни тысяч статей. Другие категории (например, <<Терминология>>) слишком узкие. Решение: комбинация широких и узких категорий с последующей фильтрацией по длине текста.
    
    \item \textbf{Обеспечение минимума в 30\,000 документов}: После фильтрации по минимальной длине (1\,000 слов) количество документов достигает ~30\,000. Это потребовало расширения списка начальных категорий и настройки глубины обхода дерева категорий до 5 уровней.
\end{enumerate}

\subsection{Выводы}

Корпус документов по тематике <<Лингвистика>> успешно собран из двух источников: $\sim$30\,000 энциклопедических статей из русскоязычной Википедии и $\sim$5\,000 научных статей из КиберЛенинки ($\sim$35\,000 документов суммарно).

\textbf{Качество корпуса:}
\begin{itemize}
    \item Тематическая однородность обеспечивается выбором релевантных категорий и фильтрацией по длине текста
    \item Минимальный размер документа (1\,000 слов) гарантирует достаточную информативность для построения поискового индекса
    \item Формат NDJSON удобен для потоковой обработки и не требует загрузки всего корпуса в память
\end{itemize}

\textbf{Преимущества двух источников:}
\begin{itemize}
    \item Энциклопедические статьи из Википедии дают широкий охват тематики
    \item Научные статьи из КиберЛенинки добавляют специализированную терминологию и академический стиль
    \item Объединённый корпус обеспечивает более полное покрытие предметной области
\end{itemize}

\textbf{Недостатки текущего подхода:}
\begin{itemize}
    \item Зависимость от структуры категорий Википедии, которая может быть неполной или устаревшей
    \item Раздел КиберЛенинки <<Языкознание и литературоведение>> включает и литературоведческие статьи, не все из которых релевантны тематике лингвистики
    \item Отсутствие контроля над качеством содержимого (возможны шаблонные или неполные статьи)
\end{itemize}

\pagebreak
