\section{Добыча корпуса документов}

\subsection{Выбор тематики}

В качестве тематики корпуса выбрана \textbf{Лингвистика} --- наука о языке, охватывающая фонетику, фонологию, морфологию, синтаксис, семантику, лексикологию, грамматику, прикладную и компьютерную лингвистику, а также смежные области (психолингвистика, социолингвистика, этимология, письменность и др.).

Русскоязычный раздел Википедии содержит обширную коллекцию статей по данной тематике, что позволяет собрать корпус требуемого объёма (более 30\,000 документов).

\subsection{Источник данных}

\textbf{Источник:} русскоязычный раздел Википедии (\url{https://ru.wikipedia.org}).

\textbf{Начальные категории для обхода} (из \texttt{scraper/config.yaml}):
\begin{itemize}
    \item Лингвистика
    \item Языки, Языкознание
    \item Фонетика, Фонология
    \item Морфология (лингвистика), Синтаксис
    \item Семантика, Лексикология
    \item Грамматика, Прикладная лингвистика
    \item Психолингвистика, Социолингвистика
    \item Когнитивная лингвистика, Лингвистическая типология
    \item Диалектология, Стилистика
    \item Этимология, Орфография
    \item Письменность, Переводоведение
    \item Терминология, Компьютерная лингвистика
\end{itemize}

\subsection{Характеристики корпуса}

\begin{table}[H]
\centering
\begin{tabular}{lr}
\toprule
\textbf{Параметр} & \textbf{Значение} \\
\midrule
Количество документов & $\sim$30\,000 \\
Формат хранения & NDJSON \\
Минимальный размер документа & 1000 слов \\
Язык & русский (с включениями на других языках) \\
\bottomrule
\end{tabular}
\caption{Характеристики корпуса документов}
\end{table}

Каждый документ содержит следующие поля:
\begin{itemize}
    \item \texttt{url} --- адрес страницы Википедии
    \item \texttt{title} --- заголовок статьи
    \item \texttt{text} --- очищенный текст статьи
    \item \texttt{word\_count} --- количество слов
    \item \texttt{paragraph\_count} --- количество параграфов
\end{itemize}

\subsection{Существующие поисковики для данного корпуса}

\begin{enumerate}
    \item \textbf{Встроенный поиск Википедии} --- ограничен только Википедией, нет возможности настраивать алгоритмы ранжирования, отсутствует явный булев поиск с операторами.
    
    \item \textbf{Google (site:ru.wikipedia.org)} --- высокое качество ранжирования, но закрытая система, невозможно изучить внутреннюю работу и алгоритмы.
    
    \item \textbf{Яндекс (site:ru.wikipedia.org)} --- аналогично Google, проприетарная реализация.
\end{enumerate}

\subsection{Недостатки существующих решений}

\begin{itemize}
    \item Невозможность настройки и изучения алгоритмов ранжирования
    \item Отсутствие булевого поиска с явными операторами AND, OR, NOT
    \item Невозможность проведения статистического анализа корпуса (закон Ципфа, частотный анализ)
    \item Закрытость реализации для учебных и исследовательских целей
    \item Ограниченный контроль над процессом токенизации и стемминга
\end{itemize}

\subsection{Журнал выполнения задания}

При выполнении задания по сбору корпуса документов были выявлены следующие проблемы и их решения:

\begin{enumerate}
    \item \textbf{Выбор подходящего объёма тематики}: Изначально рассматривались более узкие темы (например, только <<Фонетика>> или <<Морфология>>), но они не обеспечивали требуемый минимум в 30\,000 документов. Расширение до общей категории <<Лингвистика>> с включением подкатегорий (языки, фонетика, синтаксис, семантика и др.) позволило достичь целевого объёма.
    
    \item \textbf{Категории Википедии: слишком широкие или узкие}: Некоторые категории (например, <<Лингвистика>>) содержат сотни тысяч статей. Другие категории (например, <<Терминология>>) слишком узкие. Решение: комбинация широких и узких категорий с последующей фильтрацией по длине текста.
    
    \item \textbf{Обеспечение минимума в 30\,000 документов}: После фильтрации по минимальной длине (1\,000 слов) количество документов достигает ~30\,000. Это потребовало расширения списка начальных категорий и настройки глубины обхода дерева категорий до 5 уровней.
\end{enumerate}

\subsection{Выводы}

Корпус документов по тематике <<Лингвистика>> успешно собран и содержит ~30\,000 документов из русскоязычного раздела Википедии. 

\textbf{Качество корпуса:}
\begin{itemize}
    \item Тематическая однородность обеспечивается выбором релевантных категорий и фильтрацией по длине текста
    \item Минимальный размер документа (1\,000 слов) гарантирует достаточную информативность для построения поискового индекса
    \item Формат NDJSON удобен для потоковой обработки и не требует загрузки всего корпуса в память
\end{itemize}

\textbf{Покрытие тематики:}
Корпус охватывает основные аспекты лингвистики: фонетика, морфология, синтаксис, семантика, грамматика, прикладная и компьютерная лингвистика. Однако наблюдается перекос в сторону общих статей --- узкие подразделы (например, диалектология) представлены слабее.

\textbf{Потенциал для расширения:}
\begin{itemize}
    \item Добавление статей из других языковых разделов Википедии для мультиязычного поиска
    \item Включение статей о теории языка и истории лингвистики для более глубокого анализа
    \item Расширение за счёт статей о конкретных языках и диалектах
\end{itemize}

\textbf{Недостатки текущего подхода:}
\begin{itemize}
    \item Зависимость от структуры категорий Википедии, которая может быть неполной или устаревшей
    \item Отсутствие контроля над качеством содержимого статей (возможны шаблонные или неполные статьи)
    \item Ограничение только статьями Википедии исключает другие источники (книги, научные публикации)
\end{itemize}

\pagebreak
