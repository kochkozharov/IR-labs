\section*{Заключение}

В ходе выполнения лабораторной работы была реализована полнофункциональная поисковая система для корпуса статей по лингвистике из русскоязычной Википедии.

\subsection*{Выполненные задачи}

\begin{enumerate}
    \item \textbf{Добыча корпуса документов} --- собран корпус из 30\,000 статей по лингвистике. Корпус хранится в формате NDJSON.
    
    \item \textbf{Поисковый робот} --- реализован асинхронный краулер на Python с фильтрацией контента по размеру, количеству параграфов и типу страницы.
    
    \item \textbf{Токенизация} --- реализован однопроходный токенизатор на C++ с корректной обработкой UTF-8 кириллицы, фильтрацией коротких и числовых токенов.
    
    \item \textbf{Стемминг} --- реализован алгоритм Портера для русского языка, повышающий полноту поиска.
    
    \item \textbf{Закон Ципфа} --- проведён анализ распределения терминов, подтверждающий степенной закон в средней части распределения с характерными отклонениями.
    
    \item \textbf{Булев индекс} --- реализован инвертированный индекс на собственной хеш-таблице с двойным хешированием. Индекс хранится в оперативной памяти. Реализована система дампов для быстрой загрузки индекса при последующих запусках (секунды вместо минут).
    
    \item \textbf{Булев поиск} --- реализованы операторы \&\&, ||, ! (также поддерживаются AND, OR, NOT и русские операторы) с поддержкой скобок для группировки. Ранжирование по TF-IDF. Система доступна через HTTP API, полностью русскоязычный веб-интерфейс и расширенный CLI с командами :stats, :zipf, :dump.
\end{enumerate}

\subsection*{Технические решения}

\begin{itemize}
    \item Движок поиска написан на C++ без использования контейнеров STL (кроме \texttt{std::string} и \texttt{std::vector}). Хеш-таблица реализована самостоятельно с исправлением критической ошибки двойного хеширования (гарантия взаимной простоты $h_2$ с размером таблицы).
    \item Краулер реализован на Python с асинхронной загрузкой страниц.
    \item Движок работает как HTTP-сервер (библиотека \texttt{cpp-httplib}), индекс хранится в оперативной памяти. Реализована система бинарных дампов индекса для ускорения запуска.
    \item Парсинг запросов реализован через рекурсивный спуск с поддержкой скобок и правильной обработкой приоритетов операторов.
    \item Ранжирование результатов выполняется по TF-IDF для повышения релевантности.
    \item Веб-интерфейс реализован на Flask, полностью переведён на русский язык, визуализация закона Ципфа --- через Chart.js с увеличенными графиками.
    \item CLI расширен командами для статистики, визуализации и управления индексом.
    \item Все компоненты контейнеризированы (Docker Compose), лимит памяти увеличен до 16\,GB для работы с большим корпусом.
\end{itemize}

\subsection*{Критический анализ}

\textbf{Достоинства:}
\begin{itemize}
    \item Модульная архитектура с чётким разделением компонентов
    \item Эффективные собственные структуры данных ($O(1)$ поиск в хеш-таблице)
    \item Быстрый запуск благодаря системе дампов индекса (секунды вместо минут)
    \item Гибкий синтаксис запросов с поддержкой скобок и нескольких форматов операторов
    \item TF-IDF ранжирование для повышения релевантности результатов
    \item Простота развёртывания через Docker
    \item Полнофункциональный русскоязычный веб-интерфейс с визуализацией
    \item Расширенный CLI с дополнительными командами
    \item Быстрый отклик на запросы ($< 30$ мс)
\end{itemize}

\textbf{Недостатки и возможные улучшения:}
\begin{itemize}
    \item \textbf{Ранжирование}: реализован TF-IDF; можно улучшить до BM25 для учёта длины документа
    \item \textbf{Стемминг}: избыточное отсечение суффиксов; можно заменить на лемматизацию для более точного сопоставления
    \item \textbf{Индекс}: отсутствует сжатие posting lists
    \item \textbf{Поиск}: нет поддержки фразовых запросов; требуется позиционный индекс для поиска точных фраз
    \item \textbf{Нечёткий поиск}: отсутствует поддержка опечаток и вариаций написания; можно добавить fuzzy matching
    \item \textbf{Масштабируемость}: весь индекс хранится в памяти; для корпуса $> 1$М документов потребуется дисковое хранилище или распределённая архитектура
    \item \textbf{Обновление индекса}: индекс статичен; для поддержки динамических обновлений потребуется более сложная архитектура
    \item \textbf{Кэширование}: отсутствует кэширование результатов частых запросов; можно добавить для повышения производительности
\end{itemize}

\pagebreak
