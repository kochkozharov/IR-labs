\section*{Заключение}

В ходе выполнения лабораторной работы была реализована полнофункциональная поисковая система для корпуса лингвистических статей из русскоязычной Википедии.

\subsection*{Выполненные задачи}

\begin{enumerate}
    \item \textbf{Добыча корпуса документов} --- собран корпус из $\sim$35\,000 статей по лингвистике. Корпус хранится в формате NDJSON.
    
    \item \textbf{Поисковый робот} --- реализован асинхронный краулер на Python с фильтрацией контента по размеру, количеству параграфов и типу страницы.
    
    \item \textbf{Токенизация} --- реализован однопроходный токенизатор на C++ с корректной обработкой UTF-8 кириллицы, фильтрацией коротких и числовых токенов.
    
    \item \textbf{Стемминг} --- реализован алгоритм Портера для русского языка, повышающий полноту поиска на 30--40\%.
    
    \item \textbf{Закон Ципфа} --- проведён анализ распределения терминов, подтверждающий степенной закон в средней части распределения с характерными отклонениями.
    
    \item \textbf{Булев индекс} --- реализован инвертированный индекс на собственной хеш-таблице с двойным хешированием. Индекс хранится в оперативной памяти.
    
    \item \textbf{Булев поиск} --- реализованы операторы AND, OR, NOT с ранжированием по TF. Система доступна через HTTP API, веб-интерфейс и CLI.
\end{enumerate}

\subsection*{Технические решения}

\begin{itemize}
    \item Движок поиска написан на C++ без использования контейнеров STL (кроме \texttt{std::string} и \texttt{std::vector}). Хеш-таблица реализована самостоятельно.
    \item Краулер реализован на Python с асинхронной загрузкой страниц.
    \item Движок работает как HTTP-сервер (библиотека \texttt{cpp-httplib}), индекс хранится в оперативной памяти.
    \item Веб-интерфейс реализован на Flask, визуализация закона Ципфа --- через Chart.js.
    \item Все компоненты контейнеризированы (Docker Compose).
\end{itemize}

\subsection*{Критический анализ}

\textbf{Достоинства:}
\begin{itemize}
    \item Модульная архитектура с чётким разделением компонентов
    \item Эффективные собственные структуры данных ($O(1)$ поиск в хеш-таблице)
    \item Простота развёртывания через Docker
    \item Полнофункциональный веб-интерфейс с визуализацией
    \item Быстрый отклик на запросы ($< 50$ мс)
\end{itemize}

\textbf{Недостатки и возможные улучшения:}
\begin{itemize}
    \item \textbf{Ранжирование}: используется простая TF-схема; можно улучшить до TF-IDF или BM25
    \item \textbf{Стемминг}: избыточное отсечение суффиксов; можно заменить на лемматизацию
    \item \textbf{Индекс}: отсутствует сжатие; можно применить VByte или Simple9
    \item \textbf{Поиск}: нет поддержки фразовых запросов; требуется позиционный индекс
    \item \textbf{Масштабируемость}: весь индекс хранится в памяти; для корпуса $> 1$М документов потребуется дисковое хранилище
\end{itemize}

\pagebreak
