\section*{Задание}

Выполнить комплекс лабораторных работ по курсу \enquote{Информационный поиск} на оценку \enquote{удовлетворительно}. Необходимо реализовать поисковую систему для тематического корпуса текстов с веб-интерфейсом и утилитой командной строки.

\subsection*{Перечень лабораторных работ}

\begin{enumerate}
    \item \textbf{Добыча корпуса документов} --- скачать корпус, изучить его характеристики, выделить текст, найти существующие поисковики и их недостатки.
    
    \item \textbf{Поисковый робот} --- реализовать автоматический сбор документов из веб-источника с фильтрацией нерелевантного контента.
    
    \item \textbf{Токенизация} --- реализовать разбиение текстов на токены, выработать правила, описать достоинства и недостатки.
    
    \item \textbf{Стемминг} --- добавить стемминг в поисковую систему, реализовать алгоритм Портера для русского языка.
    
    \item \textbf{Закон Ципфа} --- построить график распределения терминов, наложить теоретическую кривую, объяснить расхождения.
    
    \item \textbf{Булев индекс} --- реализовать инвертированный индекс на собственных структурах данных (без STL).
    
    \item \textbf{Булев поиск} --- реализовать поиск с операторами AND, OR, NOT, создать веб-интерфейс и утилиту командной строки.
\end{enumerate}

\subsection*{Требования к реализации}

\begin{itemize}
    \item Язык программирования для поискового движка: C++ без STL (допускается \texttt{std::string} и \texttt{std::vector} для токенизации и базовой работы с данными; хеш-таблицы, деревья и прочие контейнеры реализуются самостоятельно).
    \item Для краулера и вспомогательных компонент: Python.
    \item Корпус: минимум 30\,000 документов минимум по 1000 слов.
    \item Кодировка: UTF-8.
    \item Интерфейс: веб-сервис с формой ввода и утилита командной строки.
\end{itemize}

\pagebreak
